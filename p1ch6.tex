\chapter{Виджеты диапазонов}
 Классы Gtk::Scale и Gtk:Scrollbar наследуются от класса Gtk::Range и обладают сходной функциональностью. Представляемые этими классами виджеты содержат "полосу прокрутки" и "ползунок" (иногда называемые "барабаном" в других графических окружениях). Перемещение ползунка с помощью указателя мыши приводит к его движению по полосе прокрутки, а нажатие на полосу прокрутки приводит к перемещению ползунка по направлению к точке нажатия либо в саму точку, либо на определенное расстояние в зависимости от того, какая кнопка мыши была использована. Такое поведение полосы прокрутки должно быть привычным для пользователей.

Как будет описано в разделе "Объект установки диапазона значений", каждый из виджетов диапазонов ассоциируется с объектом установки диапазона значений (Adjustment). Для изменения используемых виджетом минимального, максимального и текущего значений вам необходимо использовать методы соответствующего объекта установки диапазона значений, который вы можете получить, использовав метод класса виджета get\_adjustment(). Стандартные конструкторы виджетов диапазонов создают объекты установки диапазона значений автоматически, причем вы можете также указать существующий объект установки диапазона значений, возможно, для того, чтобы использовать его и для другого виджета. Обратитесь к разделу "Объект установки диапазона значений" для получения более подробной информации. 

