\chapter{Установка}
\section{Зависимости}
 Перед установкой gtkmm 3.0 может потребоваться установка следующих дополнительных пакетов:
\begin{itemize}
	\item libsigc++2.0
	\item GTK+3.0
	\item cairomm
	\item pangomm
	\item atkmm
\end{itemize}


Эти необходимые для работы пакеты имеют свои собственные зависимости, включающие следующие приложения и библиотеки:
\begin{itemize}
	\item pkg-config
	\item glib
	\item ATK
	\item Pango
	\item cairo
	\item gdk-pixbuf
	
\end{itemize}
\section{Unix и Linux}
\subsection{Предварительно собранные пакеты}
 Пакеты с актуальными версиями gtkmm на сегодняшний день подготовлены практически для каждого известного дистрибутива Linux. Поэтому в том случае, если вы используете Linux, вы, скорее всего, сможете начать работу с gtkmm сразу после установки пакета из официального репозитория своего дистрибутива. Дистрибутивами, поставляющими gtkmm в своих репозиториях являются Debian, Ubuntu, Red Hat, Fedora, Mandriva, Suse, а также многие другие.

Имена пакетов с gtkmm варьируются от дистрибутива к дистрибутиву (например, пакет называется libgtkmm3.0-dev в Debian и Ubuntu или gtkmm3.0-devel в Red Hat и Fedora), поэтому вам следует самостоятельно установить корректное имя пакета с помощью программы для управления пакетами из состава вашего дистрибутива и установить этот пакет таким же образом, как и любой другой. 

\subsection{Установка из исходных кодов}

 В том случае, если ваш дистрибутив не предоставляет предварительно собранного пакета с gtkmm или же в том случае, если вы хотите установить отличную версию от той, которая предоставляется вашим дистрибутивом, вы также можете установить gtkmm из исходных кодов. Исходные коды gtkmm могут быть загружены с ресурса http://gtkmm.org/.
После установки всех зависимостей следует загрузить исходные коды gtkmm, распаковать их и перейти в созданную в результате директорию. Сборка и установка gtkmm могут быть осуществлены с помощью нижеприведенной последовательности команд: 
\begin{lstlisting}[language=bash]
./configure
 make
 make install
\end{lstlisting}  
 Сценарий configure произведет проверки того, все ли необходимые программные компоненты уже установлены. Если вы не установили какие-либо из необходимых программных компонентов, он завершит свою работу и выведет сообщение об ошибке.
По умолчанию gtkmm устанавливается в директорию /usr/local. При использовании некоторых систем у вас может возникнуть необходимость в установке в другую директорию. Например, в системах Red Hat Linux вы можете использовать параметр --prefix сценария configure подобным образом: 
\begin{lstlisting}[language=bash]
./configure --prefix=/usr
\end{lstlisting}

Если вы хотите помочь с разработкой gtkmm или поэкспериментировать с новыми возможностями, вы также можете установить версию gtkmm из git. Большинству пользователей это никогда не потребуется, но все же в том случае, если вы заинтересованы в участии в процессе разработки gtkmm, можете обратиться к приложению "Работа с исходным кодом gtkmm". 
\subsection{Microsoft Windows}
 Программные компоненты GTK+ и gtkmm были спроектированы для корректной работы под управлением Microsoft Windows, при этом разработчики положительно относятся к их использованию на платформе win32. Однако, в Windows не предусмотрено стандартного механизма для установки требующихся для разработки библиотек. Пожалуйста, обратитесь к странице "Установка в Windows" для ознакомления со специфичными для платформы Windows инструкциями по установке и заметками.


